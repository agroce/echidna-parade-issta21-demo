%%
%% This is file `sample-sigconf.tex',
%% generated with the docstrip utility.
%%
%% The original source files were:
%%
%% samples.dtx  (with options: `sigconf')
%% 
%% IMPORTANT NOTICE:
%% 
%% For the copyright see the source file.
%% 
%% Any modified versions of this file must be renamed
%% with new filenames distinct from sample-sigconf.tex.
%% 
%% For distribution of the original source see the terms
%% for copying and modification in the file samples.dtx.
%% 
%% This generated file may be distributed as long as the
%% original source files, as listed above, are part of the
%% same distribution. (The sources need not necessarily be
%% in the same archive or directory.)
%%
%% The first command in your LaTeX source must be the \documentclass command.
\documentclass[sigconf]{acmart}

%%
%% \BibTeX command to typeset BibTeX logo in the docs
\AtBeginDocument{%
  \providecommand\BibTeX{{%
    \normalfont B\kern-0.5em{\scshape i\kern-0.25em b}\kern-0.8em\TeX}}}

%% Rights management information.  This information is sent to you
%% when you complete the rights form.  These commands have SAMPLE
%% values in them; it is your responsibility as an author to replace
%% the commands and values with those provided to you when you
%% complete the rights form.
\setcopyright{acmcopyright}
\copyrightyear{2021}
\acmYear{2021}
\acmDOI{10.1145/1122445.1122456}

%% These commands are for a PROCEEDINGS abstract or paper.
\acmConference[Woodstock '18]{Woodstock '18: ACM Symposium on Neural
  Gaze Detection}{June 03--05, 2018}{Woodstock, NY}
\acmBooktitle{Woodstock '18: ACM Symposium on Neural Gaze Detection,
  June 03--05, 2018, Woodstock, NY}
\acmPrice{15.00}
\acmISBN{978-1-4503-XXXX-X/18/06}


%%
%% Submission ID.
%% Use this when submitting an article to a sponsored event. You'll
%% receive a unique submission ID from the organizers
%% of the event, and this ID should be used as the parameter to this command.
%%\acmSubmissionID{123-A56-BU3}

%%
%% The majority of ACM publications use numbered citations and
%% references.  The command \citestyle{authoryear} switches to the
%% "author year" style.
%%
%% If you are preparing content for an event
%% sponsored by ACM SIGGRAPH, you must use the "author year" style of
%% citations and references.
%% Uncommenting
%% the next command will enable that style.
%%\citestyle{acmauthoryear}

%%
%% end of the preamble, start of the body of the document source.
\begin{document}

%%
%% The "title" command has an optional parameter,
%% allowing the author to define a "short title" to be used in page headers.
\title{\emph{echidna-parade}: A Tool for Diverse Multicore\\Smart Contract Fuzzing}

%%
%% The "author" command and its associated commands are used to define
%% the authors and their affiliations.
%% Of note is the shared affiliation of the first two authors, and the
%% "authornote" and "authornotemark" commands
%% used to denote shared contribution to the research.
\author{Alex Groce}
\affiliation{\institution{Northern Arizona University}\country{United States}}
\author{Gustavo Grieco}
\affiliation{\institution{Trail of Bits}\country{United States}}


%%
%% By default, the full list of authors will be used in the page
%% headers. Often, this list is too long, and will overlap
%% other information printed in the page headers. This command allows
%% the author to define a more concise list
%% of authors' names for this purpose.
\renewcommand{\shortauthors}{Groce and Grieco}

%%
%% The abstract is a short summary of the work to be presented in the
%% article.
\begin{abstract}
Echidna is a widely used fuzzer for Etherum blockchain smart
contracts.  While Echidna is an essentially single-threaded tool, it
is possible for multiple Echidna processes to communicate by use of a shared
input corpus.  Echidna provides a very large variety of configuration
options, where each smart contract may be best-tested by a non-default
configuration, and different faults of coverage targets within a
single contract may also have differing ideal configurations.  This
paper presents \emph{echidna-parade}, a tool that provides pushbutton
multicore fuzzing using Echidna as an underlying fuzzing engine, and
automatically provides sophisticated diversification of
configurations.   Even without using multiple cores, echidna-parade
can significantly improve the effectiveness of fuzzing with Echidna,
due to the advantages provided by multiple types of test configuration diversity.
\end{abstract}

\begin{CCSXML}
<ccs2012>
<concept>
<concept_id>10011007.10010940.10010992.10010998.10011001</concept_id>
<concept_desc>Software and its engineering~Dynamic analysis</concept_desc>
<concept_significance>500</concept_significance>
</concept>
<concept>
<concept_id>10011007.10011074.10011099.10011102.10011103</concept_id>
<concept_desc>Software and its engineering~Software testing and debugging</concept_desc>
<concept_significance>500</concept_significance>
</concept>
</ccs2012>
\end{CCSXML}

\ccsdesc[500]{Software and its engineering~Dynamic analysis}
\ccsdesc[500]{Software and its engineering~Software testing and debugging}

\keywords{fuzzing, smart contracts, test diversity, swarm testing,
  test length}


\maketitle

\section{Introduction}

An echidna is a spiny monotreme; Echidna is a widely used open source fuzzer for
Ethereum smart contracts~\cite{echidnaissta,echidna-code}.  The collective noun for
echidnae is \emph{``parade''}; echidna-parade is a tool for configuring and
running multiple Echidna instances to improve the
effectiveness of smart contract fuzzing.

Smart contracts for the Ethereum blockchain~\cite{buterin2013whitepaper}, most often written in Solidity~\cite{wood2014yellow}, a
JavaScript-like language, support high-value financial transactions,
as well as tracking valuable IP and even physical goods.  It is essential that
autonomous financial programs be reliable and protected against
attack.  Unfortunately, smart contracts are often \emph{neither}
correct nor secure~\cite{SurveyAttacks}.  A survey categorizing flaws
in critical contracts~\cite{FC20} estimated that fuzzing, using custom
user-defined properties, might detect more than 60\% of the most
severe, exploitable flaws in contracts, and that many of these cannot easily
be detected using purely static analysis.  Highly effective fuzzing is
therefore essential to smart contract developers and security
auditors, and the Echidna tool is used by both contract developers
(Gustavo, example?) and auditors at Trail of Bits.

Echidna is essentially a single-threaded tool that does not make
effective use of mulitple cores.  However, multiple Echidna
independent Echidna processes can be run at the same time.  An Echidna
process will, upon termination, produce a corpus of transaction
sequences needed to cover all reached contract locations and
transaction dispositions (whether the transaction succeeded or caused
a revert of the EVM) as a set of files, and an Echidna process can
take as input a set of such transaction sequences to seed fuzzing.

The \emph{echidna-parade} tool is an open source utility that orchestrates multiple
Echidna processes using this mechanism, to enable both multicore
fuzzing and more effective single-core fuzzing, by diversifying the
configuration of Echidna, in order to cover hard-to-reach code, and
discover subtle flaws. It is available via pypi ({\tt pip
    install echidna-parade}) or on GitHub
({\url{https://github.com/agroce/echidna-parade}).

\section{Basic Usage}

Using \emph{echidna-parade} is intended to be a push-button process,
requiring no additional expertise for users familiar with the Echidna
tool.  For instance, if a user used the command line:

{\scriptsize
\begin{verbatim}
> echidna-test contract.sol --config config.yaml --contract TEST
\end{verbatim}
  }

\noindent to test a smart contract, then testing the same contract with echidna-parade, using all
available CPUs for one hour, would require only a slight modification:

{\scriptsize
\begin{verbatim}
> echidna-parade contract.sol --config config.yaml --contract TEST
\end{verbatim}
  }

\section{Architecture and Diversification Strategies}

\section{Experimental Evaluation}

\section{Related Work}

\section{Conclusion}

\bibliographystyle{plain}
\bibliography{bibliography}

\end{document}