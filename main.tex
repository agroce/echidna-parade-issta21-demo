%%
%% This is file `sample-sigconf.tex',
%% generated with the docstrip utility.
%%
%% The original source files were:
%%
%% samples.dtx  (with options: `sigconf')
%% 
%% IMPORTANT NOTICE:
%% 
%% For the copyright see the source file.
%% 
%% Any modified versions of this file must be renamed
%% with new filenames distinct from sample-sigconf.tex.
%% 
%% For distribution of the original source see the terms
%% for copying and modification in the file samples.dtx.
%% 
%% This generated file may be distributed as long as the
%% original source files, as listed above, are part of the
%% same distribution. (The sources need not necessarily be
%% in the same archive or directory.)
%%
%% The first command in your LaTeX source must be the \documentclass command.
\documentclass[sigconf]{acmart}

\usepackage{code}

%%
%% \BibTeX command to typeset BibTeX logo in the docs
\AtBeginDocument{%
  \providecommand\BibTeX{{%
    \normalfont B\kern-0.5em{\scshape i\kern-0.25em b}\kern-0.8em\TeX}}}

%% Rights management information.  This information is sent to you
%% when you complete the rights form.  These commands have SAMPLE
%% values in them; it is your responsibility as an author to replace
%% the commands and values with those provided to you when you
%% complete the rights form.
\setcopyright{acmcopyright}
\copyrightyear{2021}
\acmYear{2021}
\acmDOI{10.1145/1122445.1122456}

%% These commands are for a PROCEEDINGS abstract or paper.
\acmConference[Woodstock '18]{Woodstock '18: ACM Symposium on Neural
  Gaze Detection}{June 03--05, 2018}{Woodstock, NY}
\acmBooktitle{Woodstock '18: ACM Symposium on Neural Gaze Detection,
  June 03--05, 2018, Woodstock, NY}
\acmPrice{15.00}
\acmISBN{978-1-4503-XXXX-X/18/06}


%%
%% Submission ID.
%% Use this when submitting an article to a sponsored event. You'll
%% receive a unique submission ID from the organizers
%% of the event, and this ID should be used as the parameter to this command.
%%\acmSubmissionID{123-A56-BU3}

%%
%% The majority of ACM publications use numbered citations and
%% references.  The command \citestyle{authoryear} switches to the
%% "author year" style.
%%
%% If you are preparing content for an event
%% sponsored by ACM SIGGRAPH, you must use the "author year" style of
%% citations and references.
%% Uncommenting
%% the next command will enable that style.
%%\citestyle{acmauthoryear}

%%
%% end of the preamble, start of the body of the document source.
\begin{document}

%%
%% The "title" command has an optional parameter,
%% allowing the author to define a "short title" to be used in page headers.
\title{\emph{echidna-parade}: A Tool for Diverse Multicore\\Smart Contract Fuzzing}

%%
%% The "author" command and its associated commands are used to define
%% the authors and their affiliations.
%% Of note is the shared affiliation of the first two authors, and the
%% "authornote" and "authornotemark" commands
%% used to denote shared contribution to the research.
\author{Alex Groce}
\affiliation{\institution{Northern Arizona University}\country{United States}}
\author{Gustavo Grieco}
\affiliation{\institution{Trail of Bits}\country{United States}}


%%
%% By default, the full list of authors will be used in the page
%% headers. Often, this list is too long, and will overlap
%% other information printed in the page headers. This command allows
%% the author to define a more concise list
%% of authors' names for this purpose.
\renewcommand{\shortauthors}{Groce and Grieco}

%%
%% The abstract is a short summary of the work to be presented in the
%% article.
\begin{abstract}
Echidna is a widely used fuzzer for Etherum blockchain smart
contracts that generates \emph{transaction sequences} of calls to smart contracts.  While Echidna is an essentially single-threaded tool, it
is possible for multiple Echidna processes to communicate by use of a shared
transaction sequence corpus.  Echidna provides a very large variety of configuration
options, since each smart contract may be best-tested by a non-default
configuration, and different faults or coverage targets within a
single contract may also have differing ideal configurations.  This
paper presents \emph{echidna-parade}, a tool that provides pushbutton
multicore fuzzing using Echidna as an underlying fuzzing engine, and
automatically provides sophisticated diversification of
configurations.   Even without using multiple cores, echidna-parade
can improve the effectiveness of fuzzing with Echidna,
due to the advantages provided by multiple types of test configuration
diversity.  Using \emph{echidna-parade} with multiple cores can
produce significantly better results than Echidna, in less time.
\end{abstract}

\begin{CCSXML}
<ccs2012>
<concept>
<concept_id>10011007.10010940.10010992.10010998.10011001</concept_id>
<concept_desc>Software and its engineering~Dynamic analysis</concept_desc>
<concept_significance>500</concept_significance>
</concept>
<concept>
<concept_id>10011007.10011074.10011099.10011102.10011103</concept_id>
<concept_desc>Software and its engineering~Software testing and debugging</concept_desc>
<concept_significance>500</concept_significance>
</concept>
</ccs2012>
\end{CCSXML}

\ccsdesc[500]{Software and its engineering~Dynamic analysis}
\ccsdesc[500]{Software and its engineering~Software testing and debugging}

\keywords{fuzzing, smart contracts, test diversity, swarm testing,
  test length}


\maketitle

\section{Introduction}

An echidna is a spiny monotreme; Echidna is a widely used open source fuzzer for
Ethereum smart contracts~\cite{echidnaissta}.  The collective noun for
echidnae is \emph{``parade''}; echidna-parade is a tool for configuring and
running multiple Echidna instances to improve the
effectiveness of smart contract fuzzing.

Smart contracts for the Ethereum blockchain~\cite{buterin2013whitepaper}, most often written in Solidity~\cite{wood2014yellow}, a
JavaScript-like language, support high-value financial transactions,
as well as tracking valuable IP and even physical goods.  It is essential that
autonomous financial programs be reliable and protected against
attack.  Unfortunately, smart contracts are often \emph{neither}
correct nor secure~\cite{SurveyAttacks}.  A survey categorizing flaws
in critical contracts~\cite{FC20} estimated that fuzzing, using custom
user-defined properties, might detect more than 60\% of the most
severe, exploitable flaws in contracts, and that many of these cannot easily
be detected using purely static analysis.  Highly effective fuzzing is
therefore essential to smart contract developers and security
auditors, and the Echidna tool is used by both contract developers
and internally auditors at Trail of Bits.  Echidna is, to our understanding, the most popular and best supported fuzzer for Ethereum smart contracts, with almost an order of magnitude more GitHub stars than any other open-source fuzzer.

Echidna is essentially a single-threaded tool that does not make
effective use of mulitple cores.  However, multiple Echidna
independent Echidna processes can be run at the same time.  An Echidna
process will, upon termination, produce a corpus of transaction
sequences needed to cover all reached contract locations and
transaction dispositions (whether the transaction succeeded or caused
a revert of the EVM) as a set of files, and an Echidna process can
take as input a set of such transaction sequences to seed fuzzing.

The \emph{echidna-parade} tool is an open source utility that orchestrates multiple
Echidna processes using this mechanism, to enable both multicore
fuzzing and more effective single-core fuzzing, by diversifying the
configuration of Echidna, in order to cover hard-to-reach code, and
discover subtle flaws. It is available via pypi ({\tt pip
    install echidna-parade}) or on GitHub
({\url{https://github.com/agroce/echidna-parade}).  The tool is already being used internally in security aduits by Trail of Bits.

\section{Basic Usage}

Using \emph{echidna-parade} is intended to be a push-button process,
requiring no additional expertise for users familiar with the Echidna
tool.  For instance, if a user used the command line:

{\scriptsize
\begin{verbatim}
> echidna-test contract.sol --config config.yaml --contract TEST
\end{verbatim}
  }

\noindent to test a smart contract, then testing the same contract with echidna-parade, using all
available CPUs for one hour, would require only a slight modification:

{\scriptsize
\begin{verbatim}
> echidna-parade contract.sol --config config.yaml --contract TEST
\end{verbatim}
  }

In both cases, a contract called {\tt TEST} in a file called {\tt
  contract.sol} will be tested, based on a configuration file, {\tt
  config.yaml}.  The configuration file specifices the details of the
fuzzing campaign to be run, for Echidna, and specifies a baseline for
generating a set of varying configurations, for
\emph{echidna-parade}.  The other major obvious difference is that Echidna
will run using a single core, while, if called this way, without an
{\tt --ncores} command line argument,
\emph{echidna-parade} will make use of as many cores as the tool
finds available.  Figure~\ref{fig:parade} shows output from a typical
\emph{echidna-parade} run from our experiments below.

\begin{figure*}
  {\scriptsize
    \begin{code}
Starting echidna-parade with
config=Config(files=['/Users/adg326/dss/crytic-export/flattening/DogTest.sol'],
  name='2hour.8.parade.experiment.29', resume=None, contract='DogTest',
  config=<\_io.TextIOWrapper name='plain.yaml' mode='r' encoding='UTF-8'>, bases=None, ncores=8,
  corpus\_dir='/Users/adg326/dss/corpus.2hour.8.parade.experiment.29', timeout=7200,
  gen\_time=300, initial\_time=300, seed=None, minseqLen=10,
  maxseqLen=300, PdefaultLen=0.5, PdefaultDict=0.5,
  prob=0.5, always=[])

Results will be written to: /Users/adg326/dss/2hour.8.parade.experiment.29
Identified 29 public and external functions: Dog.rely(address), Dog.deny(address), Dog.file(bytes32,address), Dog.file(bytes32,uint256),
  Dog.file(bytes32,bytes32,uint256), Dog.file(bytes32,bytes32,address), Dog.chop(bytes32), Dog.bark(bytes32,address,address), Dog.digs(bytes32,uint256),
   Dog.cage(), Vat.rely(address), Vat.deny(address), Vat.hope(address), Vat.nope(address), Vat.init(bytes32), Vat.file(bytes32,uint256), Vat.file(bytes32,bytes32,uint256),
  Vat.cage(), Vat.slip(bytes32,address,int256), Vat.flux(bytes32,address,address,uint256), Vat.move(address,address,uint256),
   Vat.frob(bytes32,address,address,address,int256,int256), Vat.fork(bytes32,address,address,int256,int256), Vat.grab(bytes32,address,address,address,int256,int256),
  Vat.heal(uint256), Vat.suck(address,address,uint256), Vat.fold(bytes32,address,int256),VowMock.fess(uint256), ClipperMock.kick(uint256,uint256,address,address)

\vspace{0.1in}
  
RUNNING INITIAL CORPUS GENERATION
- LAUNCHING echidna-test in 2hour.8.parade.experiment.29/initial blacklisting [  ] with seqLen 100 dictFr
eq 0.4 and mutConsts  [1, 1, 1, 1]

\vspace{0.1in}

SWARM GENERATION \#1: ELAPSED TIME 303.87 SECONDS / 7200
- LAUNCHING echidna-test in 2hour.8.parade.experiment.29/gen.1.0 blacklisting [ Dog.rely(address), Dog.deny(address), Dog.file(bytes32,address), Dog.file(bytes32,uint256),
  Dog.file(bytes32,bytes32,address), Vat.rely(address), Vat.deny(address), Vat.nope(address), Vat.file(bytes32,uint256), Vat.slip(bytes32,address,int256),
   Vat.move(address,address,uint256), Vat.frob(bytes32,address,address,address,int256,int256), Vat.heal(uint256) ] with seqLen 100 dictFreq 0.4 and mutConsts  [0, 3, 1000, 3]
- LAUNCHING echidna-test in 2hour.8.parade.experiment.29/gen.1.1 blacklisting [ Dog.deny(address), Dog.file(bytes32,address), Dog.file(bytes32,bytes32,uint256),
  Dog.file(bytes32,bytes32,address), Dog.cage(), Vat.hope(address), Vat.file(bytes32,bytes32,uint256), Vat.flux(bytes32,address,address,uint256),
   Vat.frob(bytes32,address,address,address,int256,int256), Vat.fork(bytes32,address,address,int256,int256), Vat.grab(bytes32,address,address,address,int256,int256),
  Vat.suck(address,address,uint256), Vat.fold(bytes32,address,int256), VowMock.fess(uint256), ClipperMock.kick(uint256,uint256,address,address) ] with seqLen 100
  dictFreq 0.88 and mutConsts  [2, 0, 2, 2000]
...
COLLECTING NEW COVERAGE: 2hour.8.parade.experiment.29/gen.1.0/corpus/coverage/-7507126444194881135.txt
COLLECTING NEW COVERAGE: 2hour.8.parade.experiment.29/gen.1.0/corpus/coverage/-6907082692337979773.txt
COLLECTING NEW COVERAGE: 2hour.8.parade.experiment.29/gen.1.2/corpus/coverage/-4795647765542071453.txt
COLLECTING NEW COVERAGE: 2hour.8.parade.experiment.29/gen.1.4/corpus/coverage/3547233554766032648.txt
COLLECTING NEW COVERAGE: 2hour.8.parade.experiment.29/gen.1.4/corpus/coverage/-7663735074641916226.txt
...
SWARM GENERATION \#2: ELAPSED TIME 612.63 SECONDS / 7200
- LAUNCHING echidna-test in 2hour.8.parade.experiment.29/gen.2.0
...
DONE!
RUNNING FINAL COVERAGE PASS...
- LAUNCHING echidna-test in 2hour.8.parade.experiment.29/coverage blacklisting [  ] with seqLen 100 dictFreq 0.4 and mutConsts  [1, 1, 1, 1]
COVERAGE PASS TOOK 62.12 SECONDS

NO FAILURES
\end{code}
}
\caption{Running \emph{echidna-parade} on the DSS Code}
\label{fig:parade}
\end{figure*}

Part of the need for \emph{echidna-parade} arises from the
complexities of the configuration file.  Echidna currently supports 34
different configuration options in this file, over 20 of which have
complex, non-boolean values.  Some of these parameters relate to the
details of the contract under test, e.g. special Ethereum addresses
with meaning to the contract, or simply determine basic properties of
a run, such as the time allowed for testing, format of the output, or
whether to try to find transactions with high gas prices or check
assertions in addition to user-defined properties.  However, there are
configurations for transaction sequence maximum length, which
functions to test, and control of low-level test generation strategies
that have a large impact on test effectiveness, and are either hard to
tune or, for that matter, impossible to ``tune'' in that different
values are essential to covering different aspects of the code or
exposing different bugs.

The \emph{echidna-parade} tool stores all results from all the separate Echidna runs it generates in separate directories, so that users can inspect failed properties and violated assertion.  The tool collects all failures and provides a summary at the end of the run, including information on where exact failure traces can be found.  Users can also use these detailed results to determine if increases in coverage or bugs were correlated with inclusion or omission of certain functions.

\begin{table*}
\centering
\begin{tabular}{l|r|r|r|r}
& Echidna (5 hours) & Parade (1 core, 5 hours) & Parade (8 cores, 1
                                                 hour) & Parade (8
                                                         cores, 2
                                                         hours) \\
  \hline
  Mean Fully Covered & 88.63 & 90.33 & 91.33 & 94.3 \\
  Median Fully Covered & 86.0 & 86.0 & 86.0 & 98.0 \\
  Std. Dev Fully Covered & 7.17 & 7.36 & 7.78 & 7.26 \\
  \hline
  Mean Non-Revert Covered & 136.2 & 138.23 & 139.7 & 143.37 \\
  Median Non-Revert Covered & 133.0 & 133.0 & 133.0 & 148.0 \\
  Std. Dev Non-Revert Covered & 9.03 & 9.36 & 9.95 & 9.14 \\                         
\end{tabular}
\caption{DSS Experiment Results}
\label{tab:exp}
\end{table*}


\section{Architecture and Diversification Strategies}

The basic architecture of \emph{echidna-parade} is simple.  It uses
the Slither static analysis tool to scan the contract under test and
extract needed information, then examines any custom Echidna
configuration options provided by the user (in the form of a YAML
file).  After this initial scan, it runs an initial run of Echidna
using default or user-configuration parameters, to form a starting
corpus of transaction sequences (API call tests, essentially).  After
this initial run, each \emph{generation} involves:

\begin{enumerate}
  \item Generating a set of novel configuations (YAML files) up to the
    number of cores the tool is allowed to use. 
    \item Spawning Echidna processess for each of these
      configurations.  Each process is seeded with the set of all
      coverage-enhancing transaction sequences found by any run thus far.
      \item Collecting any transaction sequences from these runs that
        produce new coverage, and adding them to the corpus.
        \item Reporting new corpus sequences discovered and/or new property failures to
          the user.
        \end{enumerate}

The tool allows users to configure the time alloted to the initial scan and to each generation, and initialize testing from an existing corpus.  A parade run can also be resumed, with the {\tt --resume} option, taking up where the fuzzing left off, a feature that often proves useful in fuzzers such as afl and libFuzzer.
        
The core non-bookkeeping element of the tool is the construction of the
novel configurations that provide search diversity and improve
testing.  Understanding these sources of diversity is key to
understanding the rationale for the \emph{echidna-parade} tool.

\subsection{Swarm Testing (API Call Omission)}

Swarm testing \cite{ISSTA12} is a method for improving automated test
generation that relies on identifying \emph{features} of tests, and
disabling some features in each test.  For instance, if features are
API calls, and we are testing a stack with {\tt push}, {\tt pop}, {\tt
  top}, and {\tt clear} calls, a non-swarm random test of any
significant length will normally contain multiple calls to all of
these functions.  In swarm testing, for each test some of the calls
(with probability usually equal to 0.5 for each call) will be
disabled, but different calls will be disable for each generated
test.  This produces less variance between calls within a test, but
much more variance \emph{between tests}.  Practically, in the stack
example, it will enable the size of the stack to grow much larger than
it ever would have any chance of doing in non-swarm testing, due to
some tests omitting {\tt pop} and/or {\tt clear} calls.  Swarm testing
is widely used in compiler testing
\cite{le2014compiler} and is a core element of the
testing for FoundationDB \cite{zhou2021foundationdb}.

\emph{echidna-parade} uses the Slither static analysis tool to extract
the set of public functions from tested smart contracts, and
automatically configures Echidna to omit some of these functions
(the probability of omission is 50\%, by default, but can be configured)
during each run.  We believe this provides the most important form of variation
in fuzzing.  In particular, given that some bugs and/or coverage
targets are \emph{triggered} by some function calls but
\emph{suppressed} by other calls, it is important to perform testing
with as many different call sets as possible \cite{groce2013help}.

Because users may know that \emph{some} calls are essential to the functionality being testing, \emph{echidna-parade} supports an argument, {\tt --always}, specifying function signatures that should \emph{never} be omitted from configurations.

\subsection{Test Length Variance}

Another important form of variation is the length of test sequences \cite{ASE08,ArcuriLen}.  It is
known that there is no single best choice for the length of API call
sequences in test generation; different SUTs and even different bugs
and/or coverage targets in the same SUT may ``prefer'' different test
lengths.  For example, a line of code relating to a resource limit (a
check for an array being full, for example) may require a long test to
have any chance of execution.  Another line of code may only execute
if another value has not been initialized by a different API function
that can be called.  Running a larger number of shorter tests will enable executing
this line more often, since once the code is initialized, the line can
no longer be covered in a test.  \emph{echidna-parade} therefore
varies the maximum sequence length for each Echidna run as well, in a
user-configurable way (with a bias towards the default, tuned value).

\subsection{Mutation and Search Variance}

Finally, as Holzmann et al. showed, in hard search problems in model
checking, it is extremely useful to simply vary the underlying search
methods, given the lack of an optimal solution \cite{swarmIEEE}.  The
equivalent for Echidna is to vary the sequence mutation strategies
used in the coverage-driven GA search and the frequency with which
dictionary constants mined from source code are used.
\emph{echidna-parade} therefore also varies these parameters over the
full set of valid values, but with some bias towards the default values.

\subsection{User-Controlled Variance}

In addition to these automatic variations, a user can also provide a
set of ``overlay'' YAML Echidna configuration files to
\emph{echidna-parade}, which will select one of these at random to use
in each configuration, replacing that configuration's choices for any
settings included in the chosen YAML file.  This feature makes it possible for users, with some effort, to diversify \emph{any} Echidna configuration option, with any desired probability distribution.  In particular, we expect that some users will want to use this to, e.g.,  run some fuzzing in configurations where the sender of transactions is the contract owner/creator, and some in configurations where unauthorized users originate calls into the contract.  These kinds of diversification are inherently contract-specific, unlike the generic forms of diversity \emph{echidna-parade} automatically provides.

\section{Experimental Evaluation}

We first compared the performance of Echidna and \emph{echidna-parade} on the example contract included in the \emph{echidna-parade} repo to demonstrate the value of swarm testing.  We configured both tools to use 10 minutes of CPU time (so \emph{echidna-parade} dervied no advantage in testing cycles from using multiple cores, and in fact paid a substantial overhead due to having to restart the fuzzer multiple times).  There are three detectable assertion violations in the example.  One of the violations is trivial to detect, but detecting it will also abort the trace and prevent detecting the other issues.  The other two violations require constructing very large arrays via repeated calls.  We configured \emph{echidna-parade} to always include the functions with the hard-to-find violations, and three functions known to be relevant to finding the vulnerabilities.  User knowledge of such ``key'' functions is common in real analysis of contracts, but cannot be expressed to Echidna without using \emph{echidna-parade}.

Over 10 runs, \emph{echidna-parade} detected a mean of 2.44 of the violations, with a median of two detected issues.  It detected the hardest-to-trigger violation in 7 of the runs.  Echidna alone only detected a mean of 1.6 issues, with a median detection of 1.5 issues.  Moreover, Echidna only detected the hardest-to-produce problem twice.  The difference between the two approaches was statistically significant by Mann-Whitney U-test with $p \leq 0.005$.

For a more substantial, real-world evaluation, we compared Echidna without any variance, using a fixed configuration,
to a 5 hour single-core \emph{echidna-parade} run using the default variance,
a 1 hour \emph{echidna-parade} run using 8 cores,
and a 2 hour \emph{echidna-parade} run using 8 cores.  For our evaluation, we
used the Multi Collateral Dai contract code
(\url{https://github.com/makerdao/dss}), developed as part of the
MakerDAO decentralized autonomous organization's Dai stablecoin
(market cap as we write of approximately \$3.7 billion).  The code is
a variant of the repo code used during a security audit \cite{FC20}.
Results in Table~\ref{tab:exp} show results for 30 runs of each of
these approaches.  The first set of results compares fully-covered
lines of code.  A line of code in a solidity contract is fully-covered
if it has been executed both in a context where the function in which
it resides runs to completion \emph{and} a context where the function
\emph{reverts}.  In the Ethereum blockchain, a revert causes a
call to a contract to terminate and roll back any state changes.
Testing revert behavior can be essential for detecting certain bugs.
The second comparison considers only non-reverting coverage; while
covering reverts is important, it is usually most important for
testing to run all
code in a context where the effects propagate to produce state
changes.  Despite the large overhead of continually starting up a new
echidna process, writing new coverage results to disk after each run,
and re-executing all corpus transaction sequences, even a single core
parade produces better results over 5 hours.  Making use of multiple
cores, which is not supported inside Echidna itself, a 1 or 2 hour
\emph{echidna-parade} run can produce much better results.  The 2 hour,
8-core parade run produces statistically significantly better results
for all measures, by the Mann-Whitney U test, with p-values below
0.001.  All other results, due to the notably high variance of
testing, are not significant ($0.11 \leq p \leq 0.24$).  Some coverage
targets were \emph{only} reached during a parade run; we are unsure if
these can be covered by non-diversified Echidna, in any reasonable
amount of time.  In fact, even using \emph{echidna-parade} there were coverage targets that were only covered once in total experiment runtime of over 36 CPU days!  Clearly for
effective and efficient testing of this contract, use of \emph{echidna-parade} is
required.

\section{Related Work}

Smart contract fuzzing has been a popular topic in recent literature.  To our knowledge, ContractFuzzer \cite{jiang:ase:2018} was the first such tool described in the literature.  Harvey \cite{harvey}, sFuzz \cite{sfuzz}, and other tools, in addition to Echidna \cite{echidnaissta}, the basis for our work, have subsequently appeared, and more are in development.

The foundations for the approach to diversification taken by \emph{echidna-parade} are three primary lines of research.  The first of these is the idea of swarm testing \cite{ISSTA12}, which aims to improve testing by omitting some features uniformly throughout a test, in part because a feature (e.g. API call) may \emph{suppress} some behaviors \cite{groce2013help}.   Second, there is literature showing that a single test \emph{length} may not be ideal for all SUTs or behaviors of an SUT \cite{ASE08,ArcuriLen}.  Finally, there is the swarm verification approach to model checking \cite{swarmIEEE}, a precursor to swarm testing, which diversifies search strategy parameters themselves.  The only other tool we are aware of that adds swarm testing as a layer atop an existing fuzzer is the DeepState tool \cite{goodman2018deepstate}, which was a direct inspiration for \emph{echidna-parade}.

\section{Conclusion}

Smart contract fuzzing is a challenging problem, a particularly high-stakes instance of the general API-sequence test generation problem.  As our experiments above show, high stakes contracts can be hard to explore, even with multi-hour fuzzing runs.  The \emph{echidna-parade} tool addresses this problem in two ways.  First, it adds easy multicore support to the essentially single-threaded Echidna smart contract fuzzer.  Second, \emph{echidna-parade} adds additional value by automatically diversifying the fuzzing performed by Echidna, which makes some bugs much easier to find.  The tool is open source and available via {\tt pip}, and is already being used in security audits at Trail of Bits.

\bibliographystyle{plain}
\bibliography{bibliography}

\newpage

\appendix
\section{Proposed Demonstration}

\begin{enumerate}

\item First, briefly note the importance of ethereum smart contracts, go to coindesk and etherscan pages to show the current ETH market cap, and then the massive number of transactions happening each day and associated ETH dollar value, which doesn't even cover token values where much of the action is.

\item Next show the Echidna tool page, and introduce it is as a widely used smart contract fuzzer.

\item Show a simple smart contract, such as the justlen.sol code, and then a complex one, such as part of the DSS code used in the evaluation.  Discuss the basics of Solidity, run echidna on a trivial example from the Echidna distribution, to show what a property failure and transaction sequence looks like.

\item Discuss the challenge of fuzzing complex contracts, just like any other hard fuzzing problem.  Discuss that Echidna is single-core, but can communicate via corpuses, like afl but without mid-run reading of new corpus info at present.  Discuss how many hours of Echidna will not hit all interesting targets in code like DSS.  Briefly describe use in both contract development (show pages of real contracts with Echidna tests) and in audits (mention some real bugs found in public Trail of Bits audits using Echidna).
  
\item Now show how to install echidna-parade, github repo.

\item Describe how the tool has the same interface as echidna.  Run on the simple example we used Echidna on.

\item Look at directory contents generated by echidna-parade.
  
\item Show the experimental results from the paper, discuss the advantages testing something like DSS.

\item Step through the three sources of diversity, pointing to the run on the simple example to explain swarm omission of API calls, and the differing test lengths and parameter choices.

\item Run on the justlen.sol example, which is quick enough to show during the demonstration.  Note the experimental results for it.

\end{enumerate}

\end{document}